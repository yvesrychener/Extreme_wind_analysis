\documentclass[10pt,conference,compsocconf]{IEEEtran}

\usepackage{hyperref}
\usepackage{graphicx}	% For figure environment
\usepackage{amsmath}
\usepackage{amssymb}
\usepackage{float}
\usepackage{url}
\newcommand{\beginsupplement}{%
	\setcounter{table}{0}
	\renewcommand{\thetable}{S\arabic{table}}%
	\setcounter{figure}{0}
	\renewcommand{\thefigure}{S\arabic{figure}}%
}

\begin{document}
\title{Extreme Wind Analysis}

\author{
	Matthias Minder, Yves Rychener
}

\maketitle

\begin{abstract}

\end{abstract}

\section*{Introduction} 
Modeling extreme weather events is of interest for a variety of applications, such as risk assessment for insurance companies or conception of preventive measures. In particular, questions about how often a particular event will occur, how severe events can get, and whether there are seasonal or year-dependent trends are of interest. Within the scope of this report, we will address these questions for extreme thunderstorms on a 1° longitude and 1° latitude grid cell with south-west coordinates 38° latitude and -100° longitude, situated in central Kansas. The measured quantities are the Connective Available Potential Energy ($CAPE$) and the Storm Relative Helicity ($SRH$), which have to be simultaneously high for severe thunderstorms to occur. Three-hourly time series of these events are taken into account from January 1st 1979 at 00:00 to December 31st 2014, 21:00. 
\par
In particular, we will study the value denoted $PROD$ given by $PROD = \sqrt{CAPE} \times SRH$. It captures concurrently high values of $SRH$ and $CAPE$, and is thus apt for modeling the risk of thunderstorms. Every month will be treated separately in order to capture seasonal differences. To model extreme values, a generalized extreme value distribution will be fitted to the monthly maxima of $PROD$ using maximum likelihood estimation as well as Bayesian approaches. Moreover, the generalized extreme value distribution will be fitted to the $r$-largest order statistic for every month to determine whether that improves the fit. Subsequently, we will assess dependence of $PROD$ upon time and the NINO 3.4 index, an established indicator of the El-Niño Southern Oscillation ($ENSO$). 
\par
Following this, $CAPE$ and $SRH$ are considered separately. We will study whether they are asymptotically dependent, before fitting bivariate models to the joint extremes of $CAPE$ and $SRH$. 
Finally, 50- and 100-year return levels of $PROD$ are calculated based on fitting a point process model to $PROD$ with maximum likelihood estimation, based on the Bayesian fit, and based on simulated values of $CAPE$ and $SRH$ from the bivariate model fitted before. 
\section*{Methods}


\section*{Results}


\section*{Conclusion}



%%% Bibliography
\bibliographystyle{IEEEtran}
\bibliography{literature-project}


	
\end{document}
